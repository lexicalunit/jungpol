\documentclass[14pt, a4paper, onecolumn]{extreport}
\usepackage{geometry}
\usepackage{hyperref}
\usepackage{csquotes}

\usepackage{multicol}
\usepackage{etoolbox}
\usepackage{relsize}
\patchcmd{\thebibliography}
{\list}
{\begin{multicols}{2}\smaller\list}
	{}
	{}
	\appto{\endthebibliography}{\end{multicols}}

\date{5 March 2000}

% Title Page
\title{Jungian Polarities: The Philosophical Tones and Moods Within Works by Frank Herbert}
\author{Amy Troschinetz\\
	English 3.1\\
	Mrs. C. Fitts}

\renewcommand{\baselinestretch}{1.25}

\begin{document}

\maketitle

The philosophy of Jungian Polarities states that for every truth, there is an equal but opposite truth that defines the other \cite{Hall:76}; for example, ``[d]eath is an inexorable part of the structure of life.'' \cite{McNelly}, ``[e]very angel carries a sword.'' (\underline{Dosadi} \cite{Dosadi:9}), and there cannot be good or light without evil or darkness. Frank Herbert was a life-long student of Carl G. Jung \cite{McNelly}. Frank Herbert interweaves many of Jung's theories into his works, including \underline{The Dosadi Experiment} and all of the \underline{Dune} chronicles, in order to give his works a philosophical tone/mood.

One central theme to all of Frank Herbert's books is that men/women tend to take what is powerful in their enemies and incorporate those qualities into his/her own self. As stated in \underline{Chapterhouse: Dune}, ``[w]e tend to become like \dots those we oppose'' \cite{Chapterhouse:23}.

As Herbert wrote in \underline{Dune}:

\begin{displayquote}
	To attempt an understanding of [one person] without knowing his mortal enemies \dots is to attempt seeing truth without knowing falsehood. It is the attempt to see light without knowing darkness. It cannot be. \cite{Dune:13}
\end{displayquote}

In \underline{The Dosadi Experiment}, this Jungian philosophy is ``driven to [its] ultimate expression'' when the two main characters, also enemies, become one person in order to save a planet and form their universe into something totally better \cite{Dosadi:270-271}. Through this transformation of mind and body, ``McKie'' and ``Jedrik'' become as one; therefore, to \emph{know} McKie \emph{is} to know Jedrik.

By way of this plot twist, Herbert creates a method to philosophically explore the meaning of self and of male/female. He is able to investigate and interpret the paired opposites male and female, to ``look at an object from both sides \dots'' \cite{Dosadi:272}. In doing so, he creates a single and philosophical definition of self.

Ultimately the self is a complication of all one individual's longings and those longings of his/her ancestors \cite{Berger:56}. This conclusion shows an obvious parallel to this statement from \underline{Chapterhouse: Dune}, ``[p]aired opposites define your longings and those longings imprison you'' \cite{Chapterhouse:431}.

``One cannot have a single thing without its opposite.'' \\(\underline{Chapterhouse: Dune} \cite{Chapterhouse:171}). This is true for many things, and the most obvious case/example is the American way of life. In America, we are free. But we cannot do whatever we like -- there is a controlling factor. Government and laws are that controlling factor. Not to say that freedom is only an illusion; by rather, from control comes freedom. Besides, to Frank Herbert, freedom is not so important. What is more important is ``[t]o believe you are free \dots'' (\underline{Dosadi} \cite{Dosadi:75}).

However, the American government doesn't truly control its populace, more accurately it forces discipline upon its sovereign people. It accomplishes this task through laws that, when transcended, punish the guilty to insure liberty for the others. As Frank Herbert writes, ``[s]eek freedom and become captive of your desires. Seek discipline and find your liberty'' (\underline{Children of Dune} \cite{Children:344}).

It is arguably true that too many overpowering laws damage a society. A person need only look at historical dictatorships to see the truth in this statement. Hitler's governmental methods of oppression and overpowering and controlling his people are good examples of too much law given too much power. And to say that Hitler is responsible for damaging society is a colossal understatement. ``[T]oo much law injures a society; it is the same with too little law. One seeks a balance.'' (\underline{Dosadi} \cite{Dosadi:125}). In other words, in order to effectively create new law, old law must be torn away to keep a balance (\underline{Dosadi} \cite{Dosadi:125}). Again Herbert has worked into his novel another Jungian polarity. There can be no creation of any measure without an equal measure of destruction.

This philosophy applies mostly to the structure of Law set up in the Dosadi universe created by Herbert; more specifically, ``Gowachin Law'' \cite{Dosadi:160}. Gowachin Law upholds that the ``courtareana'' should ``merely provide a convenient structure within which to hang justifications and prejudices \dots'' \cite{Dosadi:160}. However, it is impossible to make law for all possible/plausible justifications and prejudices. ``Thus, Law must be flexible, it must fit new demands. Otherwise, it becomes merely the justification of the powerful'' \cite{Dosadi:160}.

Enclosed within the symbol of The Gowachin, is a complete discussion on Law and prejudices, and it is all based on a Jungian philosophy.

Balance in nature as in all other things is accomplished by opposites. If a balance to measure mass is not evenly proportioned, one side will sink as the other rises. In order to attain balance, something must be added to the lighter side or subtracted from the heavy side. Given these two opposite actions, subtraction and addition, one final result is attained: balance. By the theory of Jungian polarities, Herbert explores the motif of balance and many others within his many works, and he does this in a philosophical manner.

In the novel \underline{Dune}, the very first line states ``[a] beginning is the time for taking care that the \textbf{balances} are correct'' \cite{Dune:3}. Through Jungian reasoning, a beginning is but a repeat of the \emph{end}. Thus, in the final lines of \underline{Chapterhouse: Dune}, the last of the Dune books, Herbert writes: ``[t]here's no secret to \textbf{balance}. You just have to feel the waves'' \cite{Chapterhouse:455}. Through this circular path/plot, Herbert introduces a philosophical element to his works; that being, ``[t]ime does not count itself'', he explains, ``you have only to look at a circle and this is apparent'' (\underline{Chapterhouse: Dune} \cite{Chapterhouse:191}).

The relevance of this idea to Jungian philosophy is clearly evident when taking into account the idea that a circle is a perfect and continuous curve, and by Jungian reasoning, the universe cannot contain such perfection without an equal amount of imperfection.

Paul Muad'Dib Atreides, a prophet in the novel \underline{Dune Messiah}, recognized this circular perfection and its consequences. Muad'Dib preaches that ``[t]here is in all things a pattern that is part of our universe.'' He goes on to say that such patterns have ``symmetery, elegance, and grace'' much as a circle does. But he also warns that ``try[ing] to copy these patterns in our lives and society,'' could result in that ``Ultimate Pattern [that] contains its own fixity'' (\underline{Dune Messiah} \cite{Messiah:380}). A society based on that fixity rather than being based on the circular repetitions of a balanced pattern, Muad'Dib concludes, could ``only move toward death'' (\underline{Dune Messiah} \cite{Messiah:380}).

The final interpretation of Muad'Dib's words contains a Jungian polarity: the ultimate perfection is death while a perfectly circular flux is the essence of all life. Overall, no interpretation of Frank Herbert's works could be complete without at least one reference to Jung and his and Herbert's philosophies on natural and balancing opposites.

\newpage

\renewcommand{\baselinestretch}{1}

\bibliography{jungpol}
\bibliographystyle{ieeetr}

\end{document}
